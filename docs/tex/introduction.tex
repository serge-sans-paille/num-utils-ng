\chapter{Introduction}
\label{chap:introduction}

\section{Contexte du projet }

La gestion des flots de donn\'ees est primordiale dans le domaine informatique. Tout d\'eveloppeur doit effectuer des calculs triviaux sur ce type de donn\'ees. Une 
biblioth\`eque de calcul num\'erique a \'et\'e cr\'e\'ee en 2002 par Suso Banderas. Elle propose dix utilitaires pour effectuer ces calculs : la biblioth\`eque num-utils \citep{numutils}.
\newline
Le projet 11 \citep{projet11}, intitul\'e << Optimisation de biblioth\`eque de calcul num\'erique >>, est  un projet informatique bas\'e sur le langage C. Il vise \`a 
am\'eliorer les performances de calcul de la biblioth\`eque num-utils, travaillant sur des flots de donn\'ees \citep{dataflux}. En effet cette derni\`ere utilise un langage interpr\'et\'e, le Perl, qui r\'eduit consid\'erablement ses performances globales. 
\newline
L'objectif de notre projet est donc de proposer une alternative \`a cette biblioth\`eque, \'ecrite en C.

\section{Probl\'ematique et approche}

Notre projet repose sur une optimisation de performances. Mais l'objectif principal reste la mise en place des outils n\'ecessaires en vue de la 
r\'ealisation et de la publication d'un logiciel dans les archives Debian \citep{debian}.
En effet, derri\`ere le probl\`eme d'optimisation se cache une autre probl\'ematique qui sera le point de d\'epart de notre projet : comment mettre
en \oe{}uvre un projet informatique ?
Notre d\'emarche consistera en premier lieu \`a d\'ecouvrir et \`a utiliser des outils n\'ecessaires \`a la r\'ealisation d'un projet informatique.
\newline
Tout d'abord une impl\'ementation des algorithmes sera fournie en langage C avec l'ensemble des outils de d\'eveloppement mis en \oe{}uvre. Ensuite, 
un paquet Debian fournissant une distribution open source sera mise en ligne. 
Outre les d\'ecouvertes techniques, cette premi\`ere approche est l'occasion de constater toute l'importance de la forme (page de manuel, readme,...) 
lors de la conception d'un logiciel informatique publi\'e. Ces \'elements facilitent l'utilisation d'un programme lors de sa diffusion publique.
\newline
La seconde \'etape du projet est l'optimisation des performances de calculs de nos programmes. Cette phase, plus courte, se veut diff\'erente de la 
premi\`ere : elle n\'ecessite avant tout un travail de r\'eflexion. C'est  un travail de fond o\`u l'objectif principal est une division par dix du temps
d'ex\'ecution des utilitaires de notre biblioth\`eque.

\section{Structure du rapport}

Nous avons s\'epar\'e le rapport technique en trois sections.
\newline
Premi\`erement, nous pr\'esentons en d\'etail la biblioth\`eque num-utils, notamment les diff\'erentes fonctionnalit\'es des dix utilitaires.
\newline
La deuxi\`eme partie concerne les outils que nous avons utilis\'es lors du d\'eveloppement de notre projet. Ces outils regroupent le gestionnaire de version, les autotools permettant d'automatiser la compilation des programmes et les utilitaires de tests. 
\newline
Dans la derni\`ere partie, nous analysons les r\'esultats des tests effectu\'es sur les dix fonctions \'ecrites en C. Nous comparons les performances de l'ancienne biblioth\`eque et de la nouvelle. Nous y expliquons \'egalement nos choix de conception.
