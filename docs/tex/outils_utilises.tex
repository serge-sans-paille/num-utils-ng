\chapter{Outils utilis\'es}
\label{chap:outils utilises}

\section{Le gestionnaire de version }
\subsection{Pr\'esentation}

La r\'ealisation de notre projet informatique a en premier lieu n\'ecessit\'e la mise en place d'une interface centralisant tous nos
documents en temps r\'eel. Pour cela nous avons utilis\'e ce que l'on appelle un gestionnaire de version. Plusieurs logiciels de gestions 
de versions existent aujourd'hui : CVS, Subversion, Mercurial, Bazar et enfin Git, ayant chacun leurs avantages et inconv\'enients. 

Ils permettent d'archiver et de conserver les diff\'erentes \'etapes de d\'eveloppement d'un projet. Ainsi il est possible de pouvoir revenir
\`a une version ant\'erieure \`a tout moment. Ils permettent \'egalement de visualiser les diff\'erences entre les r\'evisions. Cela permet un  
travail collaboratif tr\`es efficace : chaque d\'eveloppeur dispose du projet en local et peut quand il le souhaite les partager sur un serveur.

Nous avons utilis\'e pour notre projet, le logiciel Git.

\subsection{Pr\'esentation}

Git est un logiciel libre cr\'ee par Linus Torvalds, le fondateur de Linux. C'est un logiciel de version distribu\'e c'est \`a dire qu'il n'est
 pas n\'ecessaire d'utiliser un serveur pour partager nos documents : on peut chacun se synchroniser entre nous.

Au premier abord l'utilisation de git n'est pas \'evidente : toutes les manipulations se font \`a travers la console et le vocabulaire utilis\'e 
est totalement nouveau. 

Apr\`es l'installation de git, il est n\'ecessaire de s'enregistrer (mail et nom) et de cr\'eer un dossier de travail dans lequel nous 
initialisons git en utilisant la commande
\begin{verbatim} git init \end{verbatim}
Ce dossier de travail sera le d\'ep\^ot de notre projet informatique. L'ajout d'un fichier dans le dossier reconnu par git se r\'ealise
 \`a travers la commande
\begin{verbatim} git add <nom du fichier> \end{verbatim}
La commande
\begin{verbatim} git commit <nom du fichier> \end{verbatim} permet d'enregistrer la modification effectu\'ee dans le dossier de travail.
Ces 2 commandes furent utilis\'ees maintes fois lors de notre projet. 

Cependant tout cela reste local : il faudra utiliser la commande
\begin{verbatim} git push \end{verbatim}
pour rajouter au dossier centrale, contenant tous le travail des diff\'erents membres du projet, nos nouveaux fichiers. 
De m\^eme pour t\'el\'echarger ce qui fut r\'ealis\'e par les autres membres 
du projet nous avons utilis\'es la commande
\begin{verbatim} git pull \end{verbatim}

Ceci constituent les commandes de bases utilis\'es lors de notre projet. Une fois le principe assimil\'e git devient tr\`es simple d'utilisation,
 tr\`es rapide mais surtout tr\`es efficace.

Enfin une des particularit\'es de Git est l'existence de sites web collaboratifs bas\'es sur Git comme  Github ou Gitorious. 

\subsection{Pr\'esentation}

Github peut \^etre consid\'er\'e comme un r\'eseau social pour les programmeurs. En effet c'est un << entrep\^ot >> de projet utilisant git comme 
gestionnaire de version. Ainsi Github permet \`a un quelconque utilisateur d'intervenir dans un projet , d'utiliser ces codes sources etc.
GitHub cr\'ee une page de profil simple o\`u appara\^it notre nom, email, etc. Cette page affiche \'egalement notre activit\'e (commit, ajout de suivi,...)
 et nos d\'ep\^ots. De m\^eme nous pouvons recevoir si on le souhaite les mise \`a  jour ( commit, comment) li\'es \`a un projet quelconque.

Les pages d'un projet commencent par une s\'erie d'onglets permettant de parcourir les sources (page par d\'efaut), d'acc\'eder \`a la liste des
 commits, le network, les demandes de pull, les probl\`emes et les wikis.
La page source, qui est la plus importante, contient toute l'arborescence de notre projet, l'adresse SSH et HTTP et permet une vue simple
 de l'avancement de notre travail.
Les autres pages furent moins utilis\'ees car secondaire. 

Cette collaboration entre Git et Github a \'et\'e essentielle pour notre projet et a permis un suivi clair et pr\'ecis de son avancement.
\newline
\newline
\newline
\newline
ajouter dans la biblio :
\newline
https://github.com/

http://www.crunchbase.com/company/github

http://fr.wikipedia.org/wiki/GitHub

http://www.siteduzero.com/tutoriel-3-254198-gerez-vos-codes-source-avec-git.html

http://www.unixgarden.com/index.php/administration-systeme/git-it